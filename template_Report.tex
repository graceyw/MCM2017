\documentclass[]{report}


% Title Page
\title{Sustainable Cities}
\author{Dhasharath Shrivathsa, Ezra Varady, Gracey Wilson}


\begin{document}
\maketitle
\tableofcontents

\begin{abstract}
	
\end{abstract}
\section{Foreword}
TODO
\part{Motivation}
	\section{Smart Growth}
		\subsection{Factors}
		\subsection{Our Interpretation}
	
	\section{Cities}
		\subsection{Why Oakland}
		Oakland is a prime candidate for us to examine. It meets the population requirement and has publiclly committed to open data initiatives, so data that we can develop our model against is readily available.
		
		Furthermore, Oakland is a city that has a proven track record in willingness to have a discussion with individuals that have ideas as to how to make the city a better place. The city of Oakland has several problems that make solutions very appealing, such as a high crime rate, high wealth inequality, gang problems, a large poor population living in poor-quality housing and squalor.
		
		This work aims to pick apart some of the issues that Oakland faces and suggest concrete solutions to some of them.
		\subsection{Why some other non-American city}

	\section{Goals}
	Our work focuses on city park allocation, sizing, and suggested developments for the smart growth objectives of mixed land uses, creating walkable neighborhoods, preserving open space, and community development. Additionally, if one looks at how land values correlate to distance to park
	
	TODO
	
	Our second focus was trying to come up with building designs to optimally target new housing and commercial developments for the goals of creating a range of housing opportunities for groups that aren't necessarily serviced by existing developments, and making development decisions cost effective to optimally allocate housing
	
	
\part{What We Did}
	\section{Our Model}
	Our model pulls statistics out of Oakland Zoning data, real estate data and tax data to suggest where improvements can be made. Often we're interested in key statistics, such as distance to a park and price per square foot depending on area and development type.
	
	We aim to suggest where parks do the most good, and what kind of housing developments to suggest for a given desired development and location. Our model also shows how the existing development strategies are doing, and what their probable effect is.
	
	\section{Our Results}
	\section{What It Means: Our Interpretation}
	
\part{The Answer}
	\section{Our Suggestions}
	\section{Justification}
	
\end{document}          
