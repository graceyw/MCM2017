\documentclass[]{report}


% Title Page
\title{Sustainable Cities}
\author{Dhasharath Shrivathsa, Ezra Varady, Gracey Wilson}


\begin{document}
\maketitle
\tableofcontents

\begin{abstract}
		In this project, we built a model of an American city using real data in order to determine how the city might make improvements for sustainable and positive growth. We chose to study and model the city of Oakland, California because of its appropriate population size, accessible data and consistent openness to external suggestions for improvement.
	
\end{abstract}

\part{Background}
	\section{Smart Growth}
	An initiative called Smart Growth is dedicated to supporting towns and cities become more "economically prosperous, socially equitable, and environmentally	sustainable."$^{1}$ In an effort to support this initiative, our proposal for positive, sustainable growth focuses on these three qualities. 

	
	\section{City Choice}
	With a willingness to listen to external input, accessible data, and an appropriate population size, the city of Oakland, California is a prime community for us to examine.
		
	The city of Oakland has a desire to improve. With a track record of actively soliciting suggestions from local citizens and external viewpoints alike, the city is interested in facilitating discussion and collaborating on ideas for progress. This makes Oakland an ideal candidate for our efforts to focus on because they would be open to our suggestions.
	
	In an effort to achieve sustainable growth and improvement, Oakland has publicly committed to open data initiatives. This means they periodically publish their data online in the interest of both accessibility to the public and transparency. This enabled us to spend our time developing our model rather than searching for data.
		
	The city currently faces several issues. Among these are a high crime rate and a high wealth inequality, with a large number of the population living in poor-quality housing and less economically advantaged. Sonja Trauss, founder of the SF Bay Area Renter’s Federation, believes that, “The need for affordable housing in Oakland — as in San Francisco — is undeniable, Trauss says, but it should not be tackled in isolation from the rest of development.”$^{2}$ This is why we built a model that not only solves problems related to Oakland's need for housing, but also considers other areas of development.
		
	\section{Goals}
	The aim of our work is to examine issues that Oakland faces in regards to growth and present the solutions with the greatest potential impact.
	
	Our work focuses on city park allocation, sizing, and suggested developments for the smart growth objectives of mixed land uses, creating walkable neighborhoods, preserving open space, and community development. Additionally, if one looks at how land values correlate to distance to park
	
	TODO
	
	Our second focus was trying to come up with building designs to optimally target new housing and commercial developments for the goals of creating a range of housing opportunities for groups that aren't necessarily serviced by existing developments, and making development decisions cost effective to optimally allocate housing
	
	
\part{The Model}
	\section{Our Model}
	Our model pulls statistics from Oakland Zoning data, real estate data and tax data to suggest where improvements can be made. Often we're interested in key statistics, such as distance to a park and price per square foot depending on area and development type.
	
	We aim to suggest where parks have the greatest positive impact, and what kind of housing developments to suggest for a given desired development and location. Our model also shows how the existing development strategies are doing, and what their probable effect is.
	
	Using publicly available zoning information we created a heatmap of the distance to the center of open spaces for different areas in Oakland. We also found the distribution of these distances. This shows the majority of regions to be reasonably close to some sort of outdoor space. Something we hypothesized was key to creating walkable districts, as well as to give them a distinct sense of place. Several regions are further from parks, and these are places we propose focusing on. The core goal of our approach is to identify the areas where our effort would create the most impact. 
	
	\section{Our Results}
	Our results show the presence of a long tail for most metrics. This indicates that while the majority of the population is within some 'normal' bound a non-trivial population exists below this. For example though most sections of the city are fairly near a park there exist sections where the density is markedly lower, correlated with low property value. By identifying the long tail we are able to target our changes to the regions that will be most affected, and which are most in need of change. 
	
	\section{What It Means: Our Interpretation}
	
\part{The Answer}
	\section{Our Suggestions}
	\section{Justification}
	
	\section{Sources}
	
	1. https://www.epa.gov/smartgrowth/smart-growth-publication 
	
	2. http://www.citylab.com/cityfixer/2016/06/its-time-for-oakland-to-face-its-fears-and-start-building/486803/ 
	
\end{document}          